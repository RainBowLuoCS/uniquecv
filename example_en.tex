
\documentclass{uniquecv}

\usepackage{fontawesome}

% ----------------------------------------------------------------------------- %

\begin{document}

\name{Run Luo}

\medskip

\basicinfo{
  \faPhone ~ (+86) 191-7925-8153
  \textperiodcentered\
  \faEnvelope ~ lr\_8823@hust.edu.cn
  \textperiodcentered\
  \faGithub ~ github.com/run-qiao
}

% ----------------------------------------------------------------------------- %

\section{Education Background}
\dateditem{\textbf{Huazhong University of Science and Technology} \quad civil engineering \quad Undergraduate}{2019 -- 2020}
\dateditem{\textbf{Huazhong University of Science and Technology} \quad Software Engineering \quad Undergraduate}{2020 --  now\quad}
Rank:15\%(13|83) \quad GPA 3.9 \quad English:CET6


% ----------------------------------------------------------------------------- %

\section{Skills}
\smallskip
C/C++、Python、Java、Data Structure And Algorithm、Linux、Spark、Pytorch、Tensorflow、High Performance Computing Cluster And Parellel Computing


% ----------------------------------------------------------------------------- %

\section{Awards}
\datedaward{the second}{\textbf{ASC2022}}{2022.3}
\datedaward{H}{MCM}{2022.5}
\datedaward{the second of provincial}{CMC}{2021.11}
\datedaward{\small{the third}}{Mobile Application Development Competition} {2021.12}
\medskip

% ----------------------------------------------------------------------------- %

\section{Project Experience}

% ---
\datedproject{ASC2022}{Competition}{2022.1 -- 2022.3}
\textit{GPT2、Cluster Training Optimization、Parellel Computing}
\vspace{0.4ex}

Optimizing the training and convergency speed of large language model like GPT2 in multi machine nodes with multi GPU by using different acceleration training and convergence strategies
\begin{itemize}
  \item Different parallel strategies, convergence strategies and parallel frameworks are combined to make the convergence time and training time as short as possible. Making 4.7B large language model GPT2 converges in about half an hour in 4 machine nodes with 8 V100
\end{itemize}
% ---
\datedproject{SOT}{Research}{2022.3 -- 2022.5}
\textit{pytorch、SOT,\textbf{MM2022 Under Review}}
\vspace{0.4ex}

在stark基础上提出了使用高质量模板库中出现过的的相似模板提升跟踪器在当前帧的目标跟踪能力,可以很好的提高跟踪器在目标外观剧烈形变场景下的跟踪能力,增加了对齐,模板验证等结构,\textbf{在TrackingNet上的测试排行榜结果仅次于MixFormer}
% ---
\datedproject{MOT}{Research}{2021.12 -- 2022.3}
\textit{Pytorch、Attack and Defense to MOT}
\vspace{0.4ex}

针对FairMOT,ByteTracker 做对抗样本训练增强追踪器的鲁棒性,改进其后处理方法以提高多目标运动场景的追踪精度和防御能力,防御能力和运行场景下的追踪效果均有提升
% ---
\datedproject{风格变换相机}{Competition}{2021.10 -- 2021.12}
\textit{Android、Pytorch,图像风格动漫画转换,Ncnn Deployment}
\vspace{0.4ex}

android 写了一个风格变化相册,可以分享,保存,展示,使用pytorch 转onnx 转 ncnn部署在安卓端
\begin{itemize}
  \item onnx算子优化压缩网络参数,减少功耗,ncnn加速部署在手机上提高响应速度,同时集成matting网络去除背景抠图方便添加背景,转换等二次处理
\end{itemize}
% ----------------------------------------------------------------------------- %
\section{课外}
\dateditem{\textbf{Huazhong University of Science and Technology Unique Studio Club AI group}}{2021年04月 -- 至今}
\dateditem{\textbf{华中科技大学科创团队AI组组长}}{2021年08月 -- 至今}
\dateditem{\textbf{华中科技大学七边形超算队}}{2021年12月 -- 至今}
\end{document}
