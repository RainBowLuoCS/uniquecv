\documentclass{uniquecv}

\usepackage{fontawesome}

% ----------------------------------------------------------------------------- %

\begin{document}

\name{罗润}

\medskip

\basicinfo{
  \faPhone ~ (+86) 191-7925-8153
  \textperiodcentered\
  \faEnvelope ~ lr\_8823@hust.edu.cn
  \textperiodcentered\
  \faGithub ~ github.com/run-qiao
}

% ----------------------------------------------------------------------------- %

\section{教育背景}
\dateditem{\textbf{华中科技大学} \quad 土木工程 \quad 本科}{2019年 -- 2020年}
\dateditem{\textbf{华中科技大学} \quad 软件工程 \quad 本科}{2020年 --  在读\quad}
成绩:年级前15\%(13|84) \quad GPA 3.9 \quad 英语:CET6


% ----------------------------------------------------------------------------- %

\section{专业技能}
\smallskip
C/C++、Python、Java、算法与数据结构、Linux、Ray、Pytorch、Tensorflow、集群并行训练


% ----------------------------------------------------------------------------- %

\section{获奖情况}
\datedaward{二等奖}{\textbf{ASC2022}}{2022年3月}
\datedaward{\small{二等奖}}{数学竞赛}{2021年11月}
\datedaward{\small{三等奖}}{手机应用开发大赛} {2021年12月}
\datedaward{\small{H奖}}{美国大学生数学竞赛} {2022年5月}
\medskip

% ----------------------------------------------------------------------------- %

\section{项目经历}

% ---
\datedproject{超算ASC2022项目}{竞赛项目}{2022年01月 -- 2022年03月}
\textit{大语言模型,集群性能优化、并行计算训练}
\vspace{0.4ex}

优化大语言模型在多机多卡集群环境的训练速度,使用不同的加速训练和收敛策略
\begin{itemize}
  \item 组合不同的并行策略(DP,PP,TP,ZERO),收敛策略(Curriculum Learning,Post Norm,Warm Up),并行框架使收敛时间和训练时间尽可能短,8 V100卡 4.7B GPT2大语言模型半小时左右收敛
\end{itemize}
% ---
\datedproject{单目标追踪项目}{科研项目}{2022年03月 -- 2022年05月}
\textit{pytorch、单目标跟踪,\textbf{MM2022在投}}
\vspace{0.4ex}

在stark(ResNet+Transformer)基础上提出了使用高质量模板库中出现过的的相似模板(K-means)更新目标模板提升跟踪器在当前帧的目标跟踪能力,可以很好的提高跟踪器在目标外观剧烈形变场景下的跟踪能力,增加了掩码(MASK),模板验证(TAB)等结构,\textbf{在TrackingNet上的测试排行榜结果仅次于MixFormer}(MM->AAAI)
% ---
\datedproject{多目标追踪项目}{科研项目}{2021年09月 -- 2022年03月}
\textit{pytorch、多目标追踪及其攻防}
\vspace{0.4ex}

针对FairMOT(CenterNet+kalman filter+KM),ByteTracker(YOLO+kalman filter+KM)做对抗样本训练(Adversarial Training)增强追踪器的鲁棒性,改进其后处理方法以提高多目标运动场景的追踪精度和防御能力,防御能力和运行场景下的追踪效果均有提升
% ---
\datedproject{HyaDIS弹性调度}{实现项目}{2022年7月 -- present}
\textit{Ray,ColossalAI,异构,PP,DP}
\vspace{0.4ex}

在ColossalAI项目组开发HyaDIS弹性计算框架,实现大模型训练弹性分配资源,提升大集群资源利用率
\begin{itemize}
  \item PP动态释放GPU的bubble的资源给其他GPU任务,根据资源的多少动态改变batch_size,资源紧张动态启用offload将部分中间数据放入内存中动态适应模型的训练。提升集群的整体使用效率。(Ray,PP,DP,TP,ZERO)
\end{itemize}
% ---
\datedproject{风格变换相机}{竞赛项目}{2021年10月 -- 2021年12月}
\textit{Android、pytorch,图像风格动漫画转换,ncnn部署}
\vspace{0.4ex}

android 写了一个风格变化相册,可以分享,保存,展示,使用pytorch 转onnx 转 ncnn部署在安卓端
\begin{itemize}
  \item onnx算子优化压缩网络参数,减少功耗,ncnn加速部署在手机上提高响应速度,同时集成matting网络去除背景抠图方便添加背景,转换等二次处理(GAN,BG Matting)
\end{itemize}
% ----------------------------------------------------------------------------- %
\section{课外}
\dateditem{\textbf{华中科技大学联创团队AI组}}{2021年04月 -- 至今}
\dateditem{\textbf{华中科技大学科创团队AI组组长}}{2021年08月 -- 至今}
\dateditem{\textbf{华中科技大学七边形超算队}}{2021年12月 -- 至今}
\end{document}

